% Appendix A

\chapter{Test Cases for Each Module}
This section provides the test cases for each of the modules of the system developed.

\section{SNOWBALL STEMMER}
\subsection{Test Pre-requisites}
Any text in English is to be given as input. 

\section{Description}
The set of test cases to this module covers different English words with similar suffixes or roots.

\section{TEST CASES}
\subsection{TC ID: 01}
\subsubsection{Input}
Text with multiple words with the same root. 
\subsubsection{Expected Output}
All words with similar roots are stemmed to give a simplified corpus.

\subsection{TC ID: 02}
\subsubsection{Input}
Text with multiple words with the same suffix.  
\subsubsection{Expected Output}
The similar suffixes are stripped to give a simplified corpus.

\section{TOKENIZER}
\subsection{Test Pre-requisite} 
Any text in English is to be given as input. 
\subsection{Description} 
The set of test cases to this module covers different English words and special characters.
\subsection{Test Cases}
\subsubsection{TC ID: 01}
Input: Text with multiple words. 
Expected Output: String is split into the required substrings.
\subsubsection{TC ID: 02}
Input: Text with multiple words and individual special characters. 
Expected Output: String and the special characters are split into the necessary components.



\section{TF-IDF VECTORIZER}
\subsection{Test Pre-requisite}
Any simplified English text corpus, after stemming and tokenizing, is to be given as input. 
\subsection{Description} 
All test case yield a standard type of output i.e the frequency counts of words in the corpus.
\subsection{Test Cases}
\subsubsection{TC ID: 01}
Input: Simplified email corpus 
Expected Output: Vectorized weighted term associations, based on frequency.

\section{SINGLE VALUE DECOMPOSITION}
\subsection{Test Pre-requisite}
The TF-IDF matrix from the previous step is to be given as input. 
\subsection{Description}
Reduces the matrix dimensionality and extracts the important features
\subsection{Test Cases}
\subsubsection{TC ID: 01}
Input: Regular TF-IDF matrix.
Expected Output: The output is a feature-reduced TF-IDF matrix.

\section{MODIFIED DATASET CREATION}
\subsection{Test Pre-requisite}
The semantic feature se obtained from the previous step is taken as the input for this step.
\subsection{Description}
The important features are chosen to create a new dataset which is used to fit the neural network
\subsection{Test Cases} 
\subsubsection{TC ID: 01}
Input: Regular TF-IDF matrix with reduced features.
Expected Output: A new feature set is created.

\section{NEURAL NETWORK FITTING AND CLASSIFICATION}
\subsection{Test Pre-requisite}
The semantic feature se obtained from the previous step is taken as the input for this step.
\subsection{Description}
This step produces the final output by classifying the test data as spam or ham.
\subsection{Test Cases} 
\subsubsection{TC ID: 01}
Input: k-largest features chosen from the previous step
Expected Output: Correctly classified as ham (true positive case).
\subsubsection{TC ID: 02}
Input: k-largest features chosen from the previous step
Expected Output: Incorrectly classified as ham (false positive case).
\subsubsection{TC ID: 03}
Input: k-largest features chosen from the previous step
Expected Output: Correctly classified as spam (true negative case).
\subsubsection{TC ID: 04}
Input: k-largest features chosen from the previous step
Expected Output: Incorrectly classified as spam (false negative case).






