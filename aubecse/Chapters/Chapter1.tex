% Chapter 1

\chapter{INTRODUCTION} % Write in your own chapter title All Chapter headings in CAPS
\section{PROBLEM DOMAIN}
\label{S:1}

Natural Language Processing is a field of computer science and linguistics concerned with the interactions between computers and human (natural) languages. It is a process through which meaning can be extracted in the context of the sentence written.  Statistical Natural Language Processing (NLP) is a field of Natural Language Processing which uses stochastic, probabilistic and statistical methods for problem solving.It is used for processing the contents of an email using various algorithms and also eliminating the tedious job of processing unnecessary parts of the email body.
\\
Machine learning is closely related to computational statistics, which also focuses in prediction-making from training a set of data.Many users receive numerous unwanted emails each day which is why a spam filter has been created. The spam filter needs to sort incoming mail into wanted and unwanted. This can be tricky because the filter could allow too much spam into the inbox or could label some legitimate emails as spam. Machine learning can help solve this problem. The email client can be trained to learn where to put each email.A machine learning algorithm for email filtering will take in a set of labeled messages as the input and will output correct labels for the testing data.
\\
The various machine learning approaches used in spam classification are Naïve Bayesian classifier that is based on Baye’s theorem and assumes that all features are statistically independent,the SVM classifier has a  good generalization performance and its ability to handle high dimensional data by using its kernels are very effective in a wide range of bioinformatics problems.In the KNN approach emails are classified based on the class of their nearest neighbors.

Semantic similarity approach uses vector space model to represent each document which is implemented by creating the term-document matrix and a vector of email document.The latent semantic analysis uses the singular value decomposition (SVD)technique to decompose a large term-document matrix into a set of k orthogonal factors.

\section{PROBLEM STATEMENT}

Given the flooding of large unsolicited and unwarranted Spam mails circulating in the web which cause both those naive to fall into the malicious traps encapsulated in the mails and the cluttering of fake advertisements in the email-box, there is a need to create a Spam-classification system which can efficiently and intelligently classify spams thus saving the trouble of millions.

The paper proceeds to elaborate or instantiate a system that performs spam classification using both semantic similarity and the less used corpus based thesaurus approach which is implemented using a LING SPAM corpus. The generated feature space along with the input messages are used to form the input for a feed-forward neural network which classifies the input as either a Spam or a Ham.  

\section{SCOPE}

According to Commtouch’s report in first quarter of 2010, there are average 183 billion spam messages sent everyday. Such mails not only waste a lot of bandwidth, but can also cause serious damages to personal computers in the form of viruses. Spammers are getting smarter, continuously trying to come up with approaches that enable them to circumnavigate spam-filtering schemes. Furthermore, active research in emerging approaches shows that there is a continued effort to come up with better and alternative anti-spam schemes. Given that the operational landscape of spam ecosystems is continuously changing, different techniques that can be applied to old and new problem areas are being sought. Social networks, for example, have become a major breeding ground for spam-related activities. Research in these specific areas, including the behavioral model space has also intensifies with continual advancement.

\section{CONTRIBUTION}

This system uses two approaches in the semantic similarity , one of which involves the construction of a Corpus Based Thesaurus. One of the main advantages of this approach is that it  considers the contextual similarity between terms. The system can classify email as Spam or Ham with a reasonably high degree of accuracy. However, this approach is high dependent on memory, and requires heavy computational power. This system can further be refined with a more powerful system

\section{OVERVIEW OF THE THESIS}

Chapter 2 discusses the existing approaches of spam filtering in greater detail. It also analyses the advantages and disadvantages of each approach. Chapter 3 gives the requirements analysis of the system. It explains the functional and non-functional requirements, constraints and assumptions made in the implementation of the system and the various UML diagrams. 


